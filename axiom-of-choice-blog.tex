% xelatex
\documentclass[11pt,a4paper]{article}
\usepackage{geometry}
\usepackage{amsmath,amssymb,amsfonts,amsthm}
\usepackage[dvipsnames,svgnames]{xcolor}
\usepackage{enumerate}
\usepackage[inline]{enumitem}
\usepackage{thmtools}
\usepackage{fancyhdr}
\usepackage{ragged2e}
\usepackage{titling}

% \usepackage{fontspec}
% \usepackage{unicode-math}
% \setmainfont{TeX Gyre Termes}
% \setmathfont[Scale=0.93]{TeX Gyre Pagella Math}


% ---------- Preamble ----------
\pagestyle{fancyplain} % Make all pages in the document conform to the custom headers and footers
\fancyhead{} % No page header
\fancyfoot[L]{}
\fancyfoot[C]{}
\fancyfoot[R]{\thepage}
\renewcommand{\headrulewidth}{0pt} % Remove header underlines
\renewcommand{\footrulewidth}{0pt} % Remove footer underlines
\geometry{
    textwidth=160mm,
    top=25mm,
    bottom=25mm,
}
\setlist[enumerate]{
    label=\textnormal{(\arabic*)},
    leftmargin=2.5em,
    topsep=0.5em,
    itemsep=0em,
}
% \everymath{\displaystyle}


% ---------- Theorem environments ----------
\theoremstyle{plain} % default
\newtheorem{thm}{Theorem}
\newtheorem{thms}[thm]{Theorems}
\newtheorem{cor}[thm]{Corollary}
\newtheorem{prop}[thm]{Proposition}
\newtheorem{props}[thm]{Propositions}
\newtheorem{lem}[thm]{Lemma}
\newtheorem{conj}[thm]{Conjecture}
\newtheorem{quest}[thm]{Question}

\theoremstyle{definition}
\newtheorem{defn}[thm]{Definition}
\newtheorem{defns}[thm]{Definitions}

\theoremstyle{remark}
\newtheorem*{rmk}{Remark}
\newtheorem*{rmks}{Remarks}


% ---------- Symbol Macros ----------
\newcommand{\bra}[1]{\mathopen{}\left(#1\right)}
\newcommand{\sbra}[1]{\mathopen{}\left[#1\right]}
\newcommand{\cbra}[1]{\mathopen{}\left\{#1\right\}}
\newcommand{\abra}[1]{\mathopen{}\left\langle#1\right\rangle}
\newcommand{\norm}[1]{\mathopen{}\left\lVert#1\right\rVert}
\newcommand{\inp}[2]{\mathopen{}\left\langle#1,#2\right\rangle}
\newcommand{\abs}[1]{\mathopen{}\left|#1\right|}
\newcommand{\R}{\mathbb{R}}
\newcommand{\N}{\mathbb{N}}
\newcommand{\Z}{\mathbb{Z}}
\newcommand{\C}{\mathbb{C}}
\newcommand{\Q}{\mathbb{Q}}
\renewcommand{\epsilon}{\varepsilon}
\renewcommand{\phi}{\varphi}
\renewcommand{\P}{\mathcal{P}}
\DeclareMathOperator{\im}{im}


% ---------- Title Section ----------
\title{
    The Axiom of Choice: Equivalent Formulations and Surprising Consequences
}
\author{Zhijie Chen}


\begin{document}
\maketitle
\thispagestyle{empty}

What is the mysterious axiom of choice? Why the hell do we need it even just to show that every infinite set has a countable subset? Why does every vector space have a basis? \dots

\tableofcontents
\newpage


\begin{quote}
    To choose one sock from each of infinitely many pairs of socks requires the Axiom of Choice, but for shoes the Axiom is not needed.

    \hfill --- Bertrand Russell
\end{quote}

\begin{quote}
    The Axiom of Choice is obviously true; the well-ordering principle is obviously false; and who can tell about Zorn's lemma?

    \hfill --- Jerry Bona
\end{quote}

\begin{quote}
    Thought is subversive and revolutionary, destructive and terrible; thought is merciless to privilege, established institutions, and comfortable habit. Thought looks into the pit of hell and is not afraid. Thought is great and swift and free, the light of the world, and the chief glory of man.

    \hfill --- Bertrand Russell
\end{quote}



\section{Introduction}


\section{Equivalence of the Three Formulations}
\subsection{AC implies Zorn's lemma}

\subsection{Zorn's lemma implies AC}

\subsection{Zorn's lemma implies the well-ordering theorem}

\subsection{AC implies the well-ordering theorem}

\subsection{The well-ordering theorem implies AC}


\section{Consequences of the Axiom of Choice}
\subsection{Every infinite set has a countable subset.}

\subsection{Every vector space has a basis.}

\subsection{There does not exist a perfect generalization of the length of an interval to all subsets of $\R$.}

\subsection{Every nontrivial finitely generated group possesses maximal subgroups.}


\section{Afterword}


\end{document}